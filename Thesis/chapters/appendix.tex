\chapter*{Appendix}
\addcontentsline{toc}{chapter}{Appendix}
\renewcommand\thetheorem{A.\arabic{theorem}}


\begin{lemma}[Expectation of Powers of a zero-mean Gaussian random variable]
  If $X\sim\N(0,\sigma^2)$, then $\EE[X^2] = \sigma^2$ and $\EE[X^4] = 3\sigma^4$.
\end{lemma}


% \begin{theorem}[Properties of Indicator Functions]
%   \label{theorem:indiprop}
%   Let $(X,\Sigma,\mu)$ be a measure space and $A,B$ be subsets of $\Omega$, then
%   \begin{enumerate}
%     \item $\mathbf{1}_{A\cap B}=\mathbf{1}_A\mathbf{1}_B$.
%     \item $\int\limits_{X}\mathbf{1}_A\d \mu = \mu(A)$.
%   \end{enumerate}
% \end{theorem}

\begin{theorem}[Pointwise Limit of Measurable Functions is Measurable]
  \label{theorem:pointwise-limit-of-measurable-functions-is-measurable}
  Let $(f_n)$ be a sequence of measurable functions converging pointwisely to $f$. Then $f$ is measurable.
\end{theorem}

% \begin{theorem}[Approximation of a Measurable Function]
%   \label{theorem:meafuncapp}
%   Let $(X,\Sigma)$ be a measurable space and $f:\Omega\to\RR$ be a measurable function. Then there exists a sequence of step functions $(f_n)_{n\in\NN}$ that pointwise converges to $f$ i.e.
%   $$\forall x\in X, f(x)=\lim\limits_{n\to\infty}f_n(x).$$
%   If $f\ge0$, we can choose $\{f_n\}$ to be increasing.
% \end{theorem}

\begin{theorem}[Fubini's Theorem]
  \label{theorem:fubini}
  Let $(\Omega,\F,\mu)$ and $(\Gamma,\G,\nu)$ be measure spaces, and $(\Omega\times\Gamma,\F\otimes\G,\mu\times\nu)$ be the product measure space. Then for any measurable function $f:\Omega\times\Gamma\to\RR$,
  \begin{equation}  \int\limits_{\Omega\times\Gamma}f\d (\mu\times\nu) = \int\limits_{\Omega}\int\limits_{\Gamma}f\d \mu\d \nu = \int\limits_{\Gamma}\int\limits_{\Omega}f\d \nu\d \mu.
  \end{equation}
\end{theorem}

\begin{lemma}[Gronwall Inequality]
  \label{lemma:gronwall-inequality}
  Let $f:[0,T]\to\RR$ be continuous function such that
  $$f(t) \le c_1 + c_2 \int_0^t f(s)\d s.$$
  Then $f (t) \le c_1e^{c_2t}$.
\end{lemma}

% \section{General Proof of Theorem \ref{theorem:independent-expectation}}
% \label{proof:independent-expectation}

% \section{The Construction of the Itô Integral}
% \label{appendix:ito}

% \section{The Construction of the Wiener Process}
% \label{brown}
% Consider the set of all real-valued, square-integrable functions defined on $(0, 1)$ i.e. the set of $f(t)$ such that

% $$\int\limits_0^1 f^2(t)\d t<\infty.$$

% We can check effortlessly that this set is a vector space. Let us denote it by $L^2(0,1)$ and define the dot product

% $$\langle f,g\rangle = \int\limits_0^1 f(t)g(t)\d t.$$

% Now we wish to find an orthonormal basis for $L^2(0,1)$. Note that it is an infinite dimensional space. Let $\{\psi_n\}_{n=0}^\infty$ be an orthonormal basis. We must have that

% $$\int\limits_0^1 \psi_i(t)\psi_j(t)\d t = \begin{cases}
%     0,\,\,i\ne j\\
%     1,\,\,i = j
% \end{cases}.$$

% If there is another function $\varphi$ such that $\varphi$ along with $\{\psi_n\}_{n=0}^\infty$ i.e



% In fact, the functions in the orthogonal basis of $L^2(0,1)$ is the Haar functions define below.

% \begin{definition}
%     The family $\{h_k(.)\}_{k=0}^\infty$ of Haar functions is defined for $0\le t\le 1$ as follows:
%     $$h_0(t) = 1,\,\,\text{ for } 0\le t\le 1,$$
%     $$h_1(t)=\begin{cases}
%         1 &\,\,\text{ for } 0\le t\le \dfrac{1}{2}\\
%         -1 & \,\,\text{ for } \dfrac{1}{2}< t\le 1
%     \end{cases}$$
% If $2^n\le k<2^{n+1},\,n=1,2\ldots,$ we set
% $$h_k(t)=\begin{cases}
%         2^{n/2} &\,\,\text{ for } \dfrac{k-2^n}{2^n}\le t\le \dfrac{k-2^n+1/2}{2^n}\\
%         -2^{n/2} & \,\,\text{ for } \dfrac{k-2^n+1/2}{2^n}< t\le \dfrac{k-2^n+1}{2^n}\\
%         0 & \,\,\text{otherwise} 
%     \end{cases}$$
% \end{definition}

% \begin{align*}

% \end{align*}

% \begin{definition}
%     For $k=0,1,2,\ldots$, 
%     $$s_k(t)=\int\limits_{0}^t h_k(t)\d t$$
%     is the $k$th Schauder function.
% \end{definition}

% \section{Proof of Theorem \ref{theorem:score}}
% We have

% \begin{align} 
%     \dfrac{1}{2}\EE_{p_{\text{data}}}[\|\nabla_\xbf\log p(\xbf,\theta)-\nabla_\xbf\log p_{\text{data}}(\xbf)\|^2_2]
%     &=\EE_{p_{\text{data}}}\left[\dfrac{1}{2}\|\nabla_\xbf\log p_{\text{data}}(\xbf)\|^2_2+\dfrac{1}{2}\|s_\theta(\xbf)\|^2_2 - s_\theta(\xbf)^\top \nabla_\xbf\log p_{\text{data}}(\xbf)\right].  
% \end{align}


% The first term in the brackets is a constant and the second term can be computed. Hence we focus on the third term.

% \begin{align}
% \EE_{p_{\text{data}}}[-s_\theta(\xbf)^\top \nabla_\xbf\log p_{\text{data}}(\xbf)] 
% &= -\int_{\xbf\in\RR^D}s_\theta(\xbf)^\top \nabla_\xbf\log p_{\text{data}}(\xbf)p_{\text{data}}(\xbf)\d \xbf\notag\\
% &= -\int_{\xbf\in\RR^D}s_\theta(\xbf)^\top \dfrac{\nabla_\xbf p_{\text{data}}(\xbf)}{p_{\text{data}}(\xbf)}p_{\text{data}}(\xbf) \d \xbf\notag\\
% &= -\int_{\xbf\in\RR^D}s_\theta(\xbf)^\top \nabla_\xbf 
% p_{\text{data}}(\xbf)  \d \xbf
% \end{align}

% We can simplify this further by using integration by parts. Let 

% \begin{equation}
%     s_\theta(\xbf)^\top = 
%     \begin{pmatrix}
%    \varphi_1 (\xbf,\theta) & \ldots & \varphi_D (\xbf,\theta)
%     \end{pmatrix},
% \end{equation}

% \begin{equation}
%     \nabla_\xp_{\text{data}}(\xbf)  = 
%     \begin{pmatrix}
%    \dfrac{\partial p_{\text{data}}(\xbf)}{\partial x_1}\\ \vdots \\\dfrac{\partial p_{\text{data}}(\xbf)}{\partial x_D}
%     \end{pmatrix}.
% \end{equation}
% The equation becomes
% \begin{align}
% -\int_{\xbf\in\RR^D}s_\theta(\xbf)^\top \nabla_\xbf 
% p_{\text{data}} \d \xbf &= -\int_{\xbf\in\RR^D}\sum\limits_{i=1}^D \varphi_i (\xbf,\theta) \dfrac{\partial p_{\text{data}}(\xbf)}{\partial x_i} \d \xbf\notag\\
% &=\sum\limits_{i=1}^D \left[-\int_{\xbf\in\RR^D} \varphi_i (\xbf,\theta) \dfrac{\partial p_{\text{data}}(\xbf)}{\partial x_i} \d \xbf\right]
% \end{align}
% We have 

% \begin{align}
%     \dfrac{\partial \varphi_1 (\xbf,\theta) p_{\text{data}}(\xbf)}{\partial x_1} = \varphi_1 (\xbf,\theta) \dfrac{\partial p_{\text{data}}(\xbf)}{\partial x_1} +  p_{\text{data}}(\xbf)\dfrac{\partial \varphi_1 (\xbf,\theta)}{\partial x_1}.
% \end{align}

% So integrating by $x_1$ over $\RR$ gives

% \begin{equation}
% \label{equation:10}
%     \lim\limits_{x_1\to\infty} \varphi_1 (\xbf,\theta) p_{\text{data}}(\xbf) - \lim\limits_{x_1\to-\infty} \varphi_1 (\xbf,\theta) p_{\text{data}}(\xbf) = \int\limits_{-\infty}^{\infty} \varphi_1 (\xbf,\theta) \dfrac{\partial p_{\text{data}}(\xbf)}{\partial x_1}\d x_1+ \int\limits_{-\infty}^{\infty} p_{\text{data}}(\xbf)\dfrac{\partial \varphi_1 (\xbf,\theta)}{\partial x_1}\d x_1.
% \end{equation}

% Due to the assumed regularity conditions, we have

% \begin{equation}
%     \lim\limits_{x_1\to\infty} \varphi_1 (\xbf,\theta) p_{\text{data}}(\xbf)=\lim\limits_{x_1\to-\infty} \varphi_1 (\xbf,\theta) p_{\text{data}}(\xbf)=0.
% \end{equation}

% Hence (\ref{equation:10}) implies that 
% \begin{equation}
%     -\int\limits_{-\infty}^{\infty} \varphi_1 (\xbf,\theta) \dfrac{\partial p_{\text{data}}(\xbf)}{\partial x_1}\d x_1=\int\limits_{-\infty}^{\infty} p_{\text{data}}(\xbf)\dfrac{\partial \varphi_1 (\xbf,\theta)}{\partial x_1}\d x_1.
% \end{equation}

% Taking integral over $x_2,\ldots,x_D$ yields

% \begin{equation}
%     -\int_{\xbf\in\RR^D} \varphi_1 (\xbf,\theta) \dfrac{\partial p_{\text{data}}(\xbf)}{\partial x_1}\d \xbf=\int_{\xbf\in\RR^D}  p_{\text{data}}(\xbf)\dfrac{\partial \varphi_1 (\xbf,\theta)}{\partial x_1}\d \xbf.
% \end{equation}

% The calculation is similar for $i=2,\ldots,D$. Therefore, 

% \begin{align}
%     \EE_{p_{\text{data}}}[-s_\theta(\xbf)^\top \nabla_\xbf\log p_{\text{data}}(\xbf)] 
%     &= \int_{\xbf\in\RR^D}  p_{\text{data}}(\xbf)\left[\sum\limits_{i=1}^D \dfrac{\partial \varphi_i (\xbf,\theta)}{\partial x_i}\right]\d \xbf\notag\\
%     &=\EE_{p_{\text{data}}}[\mathrm{tr}(\nabla_\xs_\theta(\xbf))].
% \end{align}

% Thus we can define the loss function based on the approximating function only
% \begin{align}
%     \L(\theta) = \EE_{p_{\text{data}}}\left[\mathrm{tr}(\nabla_\xs_\theta(\xbf))+\dfrac{1}{2}\|s_\theta(\xbf)\|^2_2\right].
% \end{align}
