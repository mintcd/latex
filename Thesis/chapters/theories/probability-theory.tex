\section{Probability Theory}

Classical interpretation of probability was popular until the work of Kolmogorov in 1993 on the axiomatic probability theory \cite{kolmogorov2018foundations}. Let us take a look on classical probability theory. A random experiment is a well-defined procedure that produces an observable outcome that could not be perfectly predicted in advance. Let $\Omega$ be the set of outcomes. Let $(\omega_n)_{n=1}^N$ be the sequence of outcomes when repeating the experiment $N$ times. An event $E$ is defined as a set of some elements in $\Omega$. At the $n$-th experiment, where $n\in\{1,\ldots,N\}$, the event $E$ is said to happen if $\omega_n\in E$. Classical probability interprets the event $E$ has probability $p$ as when we repeat the experiment $N$ times, where $N$ is sufficient large, then the ratio of outcome included in $E$ over $N$ is approximately $p$. We can write this interpretation as a limit
$$\lim\limits_{N\to\infty}\dfrac{|\{\omega_n : \omega_n\in E, n\in\{1,\ldots,N\}\}|}{N} = p.$$

Classical probability theory faces difficulties in problems whose statement is ambiguous, such as Bertrand's paradox \cite{shackel2007bertrand}. Kolmogorov's axiomatic probability theory was developed from measure theory. It provides a model for each probability calculation problem. Moreover, the theory then built consistent formal constructions of probabilistic objects, such as random variables and stochastic processes.

\subsection{Probability Space}

\begin{definition}
 Let $(\Omega, \F)$ be a measurable space. A measure $\PP$ on $(\Omega, \F)$ is called a \textbf{probability measure}\index{probability measure} if $P(\Omega)=1$. The triplet $(\Omega, \F, P)$ is called a \textbf{probability space}\index{probability space}.
\end{definition}

\begin{example}
 One rolls a fair dice once. Let $\Omega=\{1,2,3,4,5,6\}$ be the set of possible outcomes. Using classical probability, one can compute probability of several events, such as
 \begin{itemize}
  \item The probability of receiving an odd-number face is $P(\{1,3,5\})=\dfrac{1}{2}$.
  \item The probability of receiving a face whose number is less that $3$ is $P(\{1,2\})=\dfrac{1}{3}$.
 \end{itemize}
 The probability space $(\Omega, \F, P)$ for this problem contains
 \begin{itemize}
  \item $\Omega=\{1,2,3,4,5,6\}$.
  \item $\F=2^\Omega$.
  \item $P(A)=\dfrac{|A|}{6},\forall A\in\F$, where $|A|$ is the number of elements in $A$.
 \end{itemize}
\end{example}

\begin{remark}
 By convention, for $A,B\in\F$, we usually write the joint probability
 $P(A\cap B)$ by $P(A,B)$.
\end{remark}

\begin{definition}
 \label{definition:property-almost-surely}
 Given a probability space $(\Omega, \F, \PP)$. A property $\P:\Omega\to\{0,1\}$ is said to hold \textbf{almost surely}\index{almost surely} with respect to $\PP$, abbreviated by $\PP$-a.s. if
 \begin{equation}
  \PP\left(\left\{\omega\in\Omega : \P(\omega)\right\}\right) = 1.
 \end{equation}
 If $\PP$ is clear, we can just say $\P$ a.s.
\end{definition}

\begin{remark}
 The definition implies
 $$P(A\cap B)=P(B|A)P(A)=P(A)P(B)$$
 and
 $$P(A|B)=\dfrac{P(A\cap B)}{P(B)}=\dfrac{P(A\cap B)}{P(B|A)}=P(A).$$
 One can also check that three formulas are equivalent. Any of them can be used to define independent events.
\end{remark}

\subsection{Random Variables}
A probability calculation problem can sometimes be solved using more than one probability space. Consider the following example.

\begin{example}
 \label{example:probability-space-map}
 Locate four students Adam, Ben, Cam and Don to four chairs $A,B,C$ and $D$ randomly. We can compute the probability that Adam sits on chair $A$ as one of the followings. The more complicated inference is that there are $4! = 24$ ways to locate the students, whereas there are $3! = 6$ locating ways such that Adam sits on chair $A$, so the probability is $\dfrac{6}{24}=\dfrac{1}{4}$. On the other hand, we can think more simply that there is no bias of any chair that Adam sits, so the probability that Adam sits on chair $A$ is $\dfrac{1}{4}$. The first probability space $(\Omega_1, \F_1, \PP_1)$ has $24$ outcomes, whereas the second one $(\Omega_2, \F_2, \PP_2)$ has $4$ outcomes.
\end{example}

The equivalence of two probability spaces in solving the problem suggests that there is a map from the probability space with larger sample space to the one with smaller sample space $X : \Omega_1 \to \Omega_2$ such that for any two semantic-equivalent events $E_1\in\F_1$ and $E_2\in\F_2$, we have $\PP_1(E_1) = \PP_2(E_2)$.

\begin{example}
 In Example \ref{example:probability-space-map}, denote by $(X,Y)$ the outcome that student $X$ sits on chair $Y$. For the first probability space, we have
 $$\Omega_1=\{
  ((\text{Adam}, A), (\text{Ben}, B), (\text{Cam}, C), (\text{Don}, D)),
  \ldots,
  ((\text{Adam}, D), (\text{Ben}, C), (\text{Cam}, B), (\text{Don}, D))\}$$
 and $P_1(E_1) = \dfrac{|E_1|}{|\Omega_1|}$ for any $E_1\subset \Omega_1$. While for the second probability space, we have
 $$\Omega_2=\{(\text{Adam}, A), (\text{Adam}, B), (\text{Adam}, C), (\text{Adam}, D)\}$$
 and $P_2(E_2) = \dfrac{|E_2|}{|\Omega_2|}$ for any $E_2\subset \Omega_2$.
 The map $X$ in this cased is defined as $X(((\text{Adam}, Y), \ldots)) = (\text{Adam}, Y)$, for any $Y\in\{A,B,C,D\}$.
\end{example}

It derives that such map $X : \Omega_2 \to \Omega_1$ is measurable. In general, such map $X$ is called a probability preserving map.

\begin{definition}
 Let $(\Omega_1, \F_1, \PP_1)$ and $(\Omega_2, \F_2, \PP_2)$ be probability spaces. A measurable map $X : \Omega_1 \to \Omega_2$ is said to be \textbf{probability preserving}\index{probability preserving map} if
 $$\PP_1(X^{-1}(E_2)) = \PP_2(E_2), \forall E_2 \in \F_2.$$
\end{definition}

\begin{theorem}
 \label{theorem:probability-measure-wrt-measurable-map}
 Let $(\Omega_1, \F_1, \PP_1)$ be a probability space and $(\Omega_2, \F_2)$ be a measurable space. For any measurable map $X : \Omega_1 \to \Omega_2$, there exists a probability measure $\PP_2$ such that $X$ is probability preserving.
\end{theorem}

\begin{proof}
 Since $X$ is measurable, for any $E_2 \in \F_2, \PP_1(X^{-1}(E_2))$. The measure $\P_2$ is established such that
 $$\PP_2(E_2) = \PP_1(X^{-1}(E_2)), \forall E_2 \in \F_2.$$
\end{proof}

For computational purposes, we select $\Omega_2$ to be a real vector space, $\F_2 = \B(\Omega)$. The map $X$ is then called a \textit{random variable}.

\begin{definition}
 \label{definition:random-variable}
 Let $(\Omega, \F, P)$ be a probability space. A function $X:\Omega\to\RR^d$ is called a $d$-dimensional \textbf{random variable}\index{random variable} if $X$ is $(\F,\B(\RR^d))$-measurable. If $X_1,\ldots, X_N$, where $N\ge 2$ are random variables, then $(X_1,\ldots,X_N)$ is called a \textbf{random vector}\index{random vector}.
\end{definition}


% New page if possible, don't let this Remark title alone
\begin{remark}
 \begin{enumerate}[label = (\roman*)]
  \item []
  \item When we mention a random variable $X$ without further specification, we implicitly assume the existence of a probability space $(\Omega, \F, \PP)$ such that $X$ is $\F$-measurable.
  \item If the range $X(\Omega)$ is countable, then $X$ is said to be a \textbf{discrete random variable}\index{discrete random variable}.
  \item As a consequence of Theorem \ref{theorem:probability-measure-wrt-measurable-map}, every random variable $X$ induces a probability measure $\PP_X: \B(\RR^d)\to\RR$ such that
        \begin{equation}
         \PP_X(B) = \PP(\{\omega\in\Omega : X(\omega)\in B\}),
        \end{equation}
        called the distribution\index{distribution} of $X$. We usually briefly write $P(X\in B):= \PP(\{\omega\in\Omega : X(\omega)\in B\})$.
 \end{enumerate}
\end{remark}

\begin{proposition}[A Random Vector is a Random Variable]
 The following functions $X_1,\ldots,X_d: \Omega\to\RR$ are random variables if and only if the function $Y=(X_1,\ldots, X_d)$ defined by
 $$Y(\omega) = (X_1(\omega),\ldots, X_d(\omega)), \forall \omega\in\Omega$$
 is a random variable.
\end{proposition}

\begin{proof}
 Suppose that $X_1,\ldots,X_d$ are random variables, then
 $$X_i^{-1}((-\infty, b])\in \F, \forall b\in \RR, i\in \{1,\ldots,d\}.$$

 Therefore,
 \begin{align*}
  Y^{-1}(\{(x_1,\ldots,x_d): x_1\le b_1,\ldots, x_d\le b_d\})
   & = \bigcap\limits_{i=1}^d X_i^{-1}((-\infty, b_i]) \in \F, \forall b_1,\ldots,b_d\in \RR.
 \end{align*}
 Thus, $Y$ is a random variable. Conversely, suppose that $Y$ is a random variable, we have
 \begin{align*}
  X_i^{-1}((-\infty, b_i])
   & = X_i^{-1}((-\infty, b_i]) \cap \Omega                                                                                                 \\
   & = X_i^{-1}((-\infty, b_i]) \cap \left(\bigcap\limits_{j\ne i}^d X_j^{-1}(\RR)\right)                                                   \\
   & = Y^{-1}\left(\{(x_1,\ldots, x_d) : x_i \le b_i \text{ and } x_j \in \RR, \forall j\ne i\}\right) \in \F, \forall i\in\{0,\ldots, d\}.
 \end{align*}
 Thus, each $X_i$ where $i\in\{0,\ldots, d\}$ is a random variable.
\end{proof}

\begin{definition}
 Let $(\Omega, \F, P)$ be a probability space and $X:\Omega\to\RR^d$ be a random variable. Then
 $$\sigma(X)=\{X^{-1}(B) : B\in \B\}$$
 is called the $\sigma$-algebra generated\index{generated $\sigma$-algebra} by $X$ (See Proposition \ref{preposition:preimage-of-a-measurable-map-over-the imaged-sigma-algebra-is-a-sigma-algebra} that $\sigma(X)$ is indeed a $\sigma$-algebra). The $\sigma$-algebra generated by a random variable $X$ is interpreted as containing on information relevant to $X$.
\end{definition}

\begin{definition}
 Let $(\Omega, \F, P)$ be a probability space and $X:\Omega\to\RR^n$ be a random variable. Then the vector in $\RR^n$
 \begin{equation}
  \EE[X]=\int_{\Omega}X\d\PP=\int_{\RR^n}x\d \PP_x
 \end{equation}
 is called the \textbf{expectation}\index{expectation} of $X$.
\end{definition}

With the notion of expectation, we can restate the $L^2$ space for random variables as
\begin{equation}
 L^2(\PP) = \left\{X : \EE[\|X\|_2^2] < \infty \right\}.
\end{equation}

\begin{definition}
 Let $X$ be a $d$-dimensional random variable. If there exists a function $p_X:\RR^n\to [0,1]$ such that
 \begin{equation}
  \label{equation:density}
  \PP(X\in B)=\int_B p_X(x)\d \xbf, \forall B\in \B(\RR^d)
 \end{equation}
 then $X$ is said to be absolutely continuous\index{absolutely continuous}. We call $p_X$ the \textbf{probability density function}\index{probability density function} (PDF) of $X$, and denote by $X\sim p_X$.
\end{definition}

\begin{remark}
 Using Proposition \ref{proposition:equivalent-borel-core}, the condition (\ref{equation:density}) is equivalent to
 \begin{equation}
  \PP(x_1\le b_1, \ldots, x_n\le b_n)=\int\limits_{-\infty}^{b_d}\ldots\int\limits_{-\infty}^{b_1} p_X(x_1,\ldots,x_d)\d x_1\ldots\d x_d, \forall (b_1,\ldots,b_d)\in \RR^d.
 \end{equation}
 Therefore, in later proofs we will simply take the integral in Borel sets $B=\{(x_1,\ldots,x_d): x_i\le b_i, i=1,\ldots,d\}$.
\end{remark}

\begin{example}
 A ubiquitous distribution used probability theory and statistics is the Gaussian distribution $\N(\bm{\mu}, \Sigma)$. The normal density is given by
 \begin{equation}
  \N(\xbf;\bm{\mu}, \Sigma) := \N(\bm{\mu}, \Sigma)(\xbf) = \dfrac{1}{\sqrt{(2\pi)^d\det(\Sigma)}}\exp\left\{-\dfrac{1}{2}(\xbf-\bm{\mu})(\xbf-\bm{\mu})^\top\right\}.
 \end{equation}
\end{example}

\begin{remark}
 When we say $\{\xbf_n\}_{n=1}^N$ is sample from a PDF $p_X$, we mean that
 $$\lim\limits_{N\to\infty}\dfrac{|\{n: \xbf_n\in B\}|}{N} = \int\limits_{B}p_X(\xbf)\d\xbf,$$
 for any $B\in\B(\RR^d)$.
\end{remark}

If a random variable $X$ is discrete, then the function $p_X:\RR^d\to[0,1]$, such that $p_X(x) = \PP_X(x)$ is called the probability mass function (PMF) of $X$. Discreteness and absolute continuity are two extreme class of random variables. In such cases, the probability density function or mass function, if exists, contains all information about $X$. We do not concern discrete random variables in our scope. For absolutely continuous random variables, we define the joint density and the marginal density as an analogy to events.

\begin{definition}
 Let $X:\Omega\to\RR^{d_X}$ and $Y:\Omega\to\RR^{d_Y}$ are absolutely continuous random variables with density $p_X$ and $p_Y$, respectively. A function $p_{XY}:\RR^{d_X}\times \RR^{d_Y}\to[0,1]$ is called the \textbf{joint PDF}\index{joint PDF} if
 \begin{equation}
  \int_B\int_{A}p_{XY}(x,y)\d\xbf\d\ybf = \PP(\xbf\in A, \ybf\in B),\forall A\in\B(\RR^{d_X}), B\in\B(\RR^{d_Y}).
 \end{equation}
 A function $p_{X|Y}:\RR^{d_X} \times \RR^{d_y} \to[0,1]$ is called a \textbf{conditional PDF}\index{conditional PDF} if
 \begin{equation}
  \int_B \int_{A}p_{X|Y}(x, y) p_Y(y) \d\xbf \d\ybf  = \PP(\xbf\in A , \ybf\in B), \forall A\in\B(\RR^{d_X}), B\in\B(\RR^{d_Y}).
 \end{equation}
\end{definition}

\begin{theorem}[Algebra of density functions]
 \label{theorem:algebra-of-density-functions}
 Let $X$ and $Y$ be random variables with PDFs $p_X$ and $p_X$, respectively.
 \begin{enumerate}[label=(\roman*), ref=(\roman*)]
  \item The joint PDF of $X$ and $Y$ is given by
        \begin{equation}
         p_{X,Y}(x,y)=\dfrac{\partial^2 P_{X,Y}(x,y)}{\partial x\partial y},
        \end{equation}
        where $P_{X,Y}$ be the CDF of the random variable $(X,Y)$.
  \item Assume that $p_Y$ takes only positive values. The conditional PDF of $X$ given $Y$ is
        \begin{equation}
         p_{X|Y}(x,y)=\dfrac{p_{X,Y}(x,y)}{p_Y(y)}.
        \end{equation}
 \end{enumerate}
\end{theorem}
\begin{proof}
 Consider
 $$A=\{(x_1,\ldots,x_d): x_i\le a_i, i=1,\ldots,d\} \text{ and } B=\{(x_1,\ldots,x_d): x_i\le b_i, i=1,\ldots,d\}.$$
 The proof is derived from the Fundamental Theorem of Calculus.
\end{proof}

\begin{remark}
 If there is no ambiguity, we usually abbreviate the expressions
 $$p_{X,Y}(x\in A,y\in B) := p(x\in A,y\in B) \text{ and } p_{X|Y}(x\in A,y\in B) := p(x\in A|y\in B).$$
\end{remark}

Along with equality and convergence defined in $L^2$ space, almost sure equality and convergence are applicable for random variables as a stronger comparing criterion.

\begin{definition}
 Let $(\Omega, \F, P)$ be a probability space. Two random variables $X$ and $Y$ are said to be equal almost surely if
 $$\PP(\{\omega\in\Omega : X(\omega)=Y(\omega)\})=1,$$
 denoted by $X=Y, \,\,a.s.$
\end{definition}

\begin{definition}
 Let $(\Omega, \F, P)$ be a probability space. A sequence $(X_n)_{n\in \NN}$ of random variables is said to converge almost surely to a random variable $X$ if
 $$\PP\left(\left\{\omega\in\Omega : \lim\limits_{n\to\infty}X_n(\omega)=X(\omega)\right\}\right)=1.$$
\end{definition}

We can approximate how two random variables $X$ and $Y$ behave similarly to each other by their PDFs. In such case, we define a divergence\index{divergence}. A divergence is not a metric, but it gives more information than equality in distribution. Here we introduce the popular Kullback–Leibler divergence, which used in diffusion models (see Section \ref{section:diffusion-models}).

\begin{definition}
 \label{definition:KL-divergence}
 Let $P,Q$ be distribution functions. Assume that the corresponding densities $p$ and $q$ exist. The \textbf{Kullback–Leibler divergence}\index{Kullback–Leibler divergence} from $P$ to $Q$ is defined as
 \begin{equation}
  D_{\mathrm{KL}}(P\|Q) = \int_{\RR^d} p(x)\ln\dfrac{p(x)}{q(x)}\d x.
 \end{equation}
\end{definition}

\begin{remark}
 We can check that $D_{\mathrm{KL}}(P\|Q)\ge0$ with equality holds if $p=q$. In general, KL divergence is not symmetric i.e. $D_{\mathrm{KL}}(P\|Q)\ne D_{\mathrm{KL}}(Q\|P).$ Each expression serves a different purpose \cite{bishop2006pattern}.

 The symmetrized version developed from KL divergence called Jensen-Shannon (JS) divergence is also applicable
 \begin{equation}
  D_{\mathrm{JS}}(P\|Q) = \dfrac{1}{2}\left(D_{\mathrm{KL}}(P\|Q) + D_{\mathrm{KL}}(Q\|P)\right).
 \end{equation}
\end{remark}

\begin{example}
 Let $P:\left(a:\dfrac{3}{5}, b:\dfrac{1}{5}, c:\dfrac{1}{5}\right)$ is a categorical distribution. Here we mean that $P(x=a)=\dfrac{3}{5}$ and so on. Also let $Q:\left(a:\dfrac{1}{9}, b:\dfrac{4}{9}, c:\dfrac{4}{9}\right)$. Then
 $$D_{\mathrm{KL}}(P\|Q)=\dfrac{3}{5}\ln\left(\dfrac{3}{5}\cdot \dfrac{1}{9}\right)+\dfrac{1}{5}\ln\left(\dfrac{1}{5}\cdot \dfrac{4}{9}\right)+\dfrac{1}{5}\ln\left(\dfrac{1}{5}\cdot \dfrac{4}{9}\right)\approx 0.692,$$
 $$D_{\mathrm{KL}}(Q\|P)=\dfrac{1}{9}\ln\left(\dfrac{1}{9}\cdot \dfrac{3}{5}\right)+\dfrac{4}{9}\ln\left(\dfrac{4}{9}\cdot \dfrac{1}{5}\right)+\dfrac{4}{9}\ln\left(\dfrac{4}{9}\cdot \dfrac{1}{5}\right)\approx 0.522.$$
\end{example}

\begin{example}
 Let $p=\N(0,1)$ and $q=\N(1,1)$. Then
 \begin{align*}
  D_{\mathrm{KL}}(P\|Q)
   & =\int\limits_{-\infty}^\infty p(x)\ln\dfrac{p(x)}{q(x)}\d x                                                                                                                                                                \\
   & =\int\limits_{-\infty}^\infty \dfrac{1}{\sqrt{2\pi}}\exp\left\{-\dfrac{1}{2}x^2\right\}\ln\dfrac{\frac{1}{\sqrt{2\pi}}\exp\left\{-\frac{1}{2}x^2\right\}}{\frac{1}{\sqrt{2\pi}}\exp\left\{-\frac{1}{2}(x-1)^2\right\}}\d x \\
   & =\int\limits_{-\infty}^\infty \dfrac{1}{\sqrt{2\pi}}\exp\left\{-\dfrac{1}{2}x^2\right\}\left(-\dfrac{1}{2}x^2+\dfrac{1}{2}(x-1)^2\right)\d x                                                                               \\
   & =\int\limits_{-\infty}^\infty \dfrac{1}{2\sqrt{2\pi}}\exp\left\{-\dfrac{1}{2}x^2\right\}\left(1-2x\right)\d x                                                                                                              \\
   & =\dfrac{1}{2}\int\limits_{-\infty}^\infty\dfrac{1}{\sqrt{2\pi}}\exp\left\{-\dfrac{1}{2}x^2\right\}\d x-\dfrac{1}{\sqrt{2\pi}}\int\limits_{-\infty}^\infty\exp\left\{-\dfrac{1}{2}x^2\right\}\cdot x\d x                    \\
   & =\dfrac{1}{2},
 \end{align*}
 since $\exp\left\{-\dfrac{1}{2}x^2\right\}\cdot x$ is an odd function.
\end{example}

\subsection{Independence of Events and Random Variables}
\begin{definition}[Independent Events]
 Let $(\Omega, \F,\PP)$ be a probability space. Two events $A,B\in\F$ are said to be independent if
 \begin{equation}
  \PP(A \cap B)=\PP(A)\PP(B).
 \end{equation}
\end{definition}

\begin{theorem}[Conditional probability measure]
 Let $(\Omega, \F,\PP)$ be a probability space and an event $E\in\F$ such that $\PP(E)>0$. Define a function $\PP(\cdot|E):\F\to\RR$ as
 \begin{equation}
  \PP(X|E) = \dfrac{\PP(X\cap E)}{\PP(E)}
 \end{equation}
 Then $\PP(\cdot|E)$ is a probability measure, called $\PP(\cdot|E)$ the probability conditioned on an event $E$.
\end{theorem}

\begin{proof}
 For any $X\in\F$, we have $X\cap E\subset X$. Hence, $P(X,A)\le P(A)$, implying that $0\le P(X|A)\le 1.$

 On the other hand,
 $$\PP(\varnothing | E)=\dfrac{P(\varnothing\cap E)}{\PP(E)} = \dfrac{P(\varnothing)}{\PP(E)}=0 \text{ and } \PP(\Omega | E) =\dfrac{P(\Omega\cap E)}{\PP(E)} = \dfrac{\PP(E)}{\PP(E)} = 1.$$

 Finally, for a pairwise disjoint collection $\{X_i\}_{i=1}^\infty$, the collection $\{X_i\cap A\}_{i=1}^\infty$ is also pairwise disjoint. Therefore,
 \begin{align*}
  \sum\limits_{i=1}^\infty P(X_i|A)
   & = \dfrac{1}{P(A)}\sum\limits_{i=1}^\infty P(X\cap A)                  \\
   & =\dfrac{1}{P(A)}P\left(\bigcup\limits_{i=1}^\infty (X_i\cap A)\right) \\
   & =\dfrac{1}{P(A)}P\left(\bigcup\limits_{i=1}^\infty X_i \cap A \right) \\
   & = P\left(\bigcup\limits_{i=1}^\infty X_i|A\right).
 \end{align*}
 Thus, $\PP(\cdot|E)$ is a probability measure.
\end{proof}

\begin{remark}
 One can check that the independence between events $A$ and $B$ can be equivalently defined by one of the following formulas
 \begin{enumerate}[label=(\arabic*)]
  \item $\PP(A \cap B)=\PP(A)\PP(B)$,
  \item $\PP(A | B)=\PP(A)$,
  \item $\PP(B | A)=\PP(B)$.
 \end{enumerate}
\end{remark}

\begin{definition}
 Two random variables $X:\Omega\to\RR^{d_X}$ and $Y:\Omega\to\RR^{d_Y}$ are said to be independent if
 \begin{equation}
  \label{definition:independent-random-variables}
  \PP(X\in B_X, Y\in B_Y)=\PP(X\in B_X)\PP(Y\in B_Y), \forall B_X\in\B(\RR^{d_X}), B_Y\in\B(\RR^{d_Y}).
 \end{equation}
\end{definition}

\begin{proposition}
 Let $X:\Omega\to\RR^{d_X}$ and $Y:\Omega\to\RR^{d_Y}$ be absolutely continuous random variables. Then $X$ and $Y$ are independent if either
 \begin{enumerate}[label=(\arabic*)]
  \item $p_{XY}=p_Xp_Y$,
  \item $p_{X|Y}=p_X$,
  \item $p_{Y|X}=p_Y$.
 \end{enumerate}
\end{proposition}

\begin{proposition}
 \label{theorem:independent-expectation}
 Let $X$ and $Y$ be absolutely continuous $L^1$ real-value random variables. If $X$ and $Y$ are independent, then
 \begin{equation}
  \EE[XY]=\EE[X]\EE[Y].
 \end{equation}
\end{proposition}

\begin{proof}
 We have
 \begin{align*}
  \EE[XY]
   & = \int_{\RR}\int_{\RR}p_{X,Y}(x,y)\d x\d y  \\
   & = \int\_{\RR}\int_{\RR}p_X(x)p_Y(y)\d x\d y \\
   & =\int_{\RR}p_X(x)\d x\int_{\RR}p_Y(y)\d y   \\
   & =\EE[X]\EE[Y].
 \end{align*}
\end{proof}

\subsection{Conditional Expectations}
The concept of conditional expectations generalizes the study of expectation and independence.

\begin{definition}[Expectation Conditioned on an Event]
 Let $(\Omega, \F,\PP)$ be a probability space and an event $E\in\F$. The conditional expectation of $X$ given $E$ is defined as
 \begin{equation}
  \EE(X|E)=\int_{E}X\d \PP(X|E).
 \end{equation}
\end{definition}

\begin{example}
 Three coins, 10p, 20p and 50p are tossed. The values of those coins that land heads up are added to work out the total amount $X$. Calculate the expected total amount given that two coins have landed heads up.
\end{example}

\begin{solution}
 Let $E$ denote the even that two coins have landed heads up, then $E=\{HHT, HTH, THH\}$. Each element has the same probability of $\dfrac{1}{8}$ so $P(E)=\dfrac{3}{8}$. Therefore,

 $$P(X|E)=\dfrac{1}{P(E)}\left(\dfrac{1}{8}(10+20)+\dfrac{1}{8}(10+50)+\dfrac{1}{8}(20+50)\right)=53\dfrac{1}{3}.$$
\end{solution}

The expectation conditioned on a random variable and the expectation conditioned on a $\sigma$-algebra is closely related. Intuitively, let $X$ and $Y$ be random variables. The expectation of $X$ given $Y$ is a random variable $Z := \EE[X|Y]$ such that $\EE[Z]$ is the best guess for the value of $X$ given all possible occurrences of $Y$. However, formal definition of $\EE[X|Y]$ concerns about $\sigma(Y)\subset\F$. Generally, consider a sub-$\sigma$-algebra $\G$ of $\F$. Because $X$ may not be $\G$-measurable, when just looking at $\G$, we may lack information about $X$. The expectation of $X$ given $\G$ is thought of as a random variable $T:= \EE[X|\G]$ such that $\EE[T]$ is the best guess for the value of $X$ when seeing $\G$.

\begin{definition}[Expectation Conditioned on a $\sigma$-algebra]
 Let $(\Omega, \F, \PP)$ be a probability space and $\G$ be a sub-$\sigma$-algebra of $\F$. The conditional expectation a random variable of $X$ conditioned on $\G$ is a $\G$-measurable random variable $\EE(X|\G)$ such that
 \begin{equation}
  \int_A \EE(X|\G)\d \PP=\int_A X\d \PP, \forall A\in\G.
 \end{equation}
\end{definition}

\begin{definition}[Expectation Conditioned on a Random Variable]
 Let $(\Omega, \F, \PP)$ be a probability space and $X,Y$ be random variables. The expectation of $X$ conditioned on $Y$ a $\sigma(Y)$-measurable random variable $\EE(X|Y)$ such that
 \begin{equation}
  \int_A \EE(X|Y)\d \PP = \int_A X\d \PP, \forall A\in\sigma(Y).
 \end{equation}
\end{definition}

The next theorem indicates that the conditional expectation depends only on the $\sigma$-algebra generated by a random variable, but not the random variable itself.

\begin{theorem}
 Let $Y,Z$ be random variables such that $\sigma(Y)=\sigma(Z)$. Then for any random variable $X$,
 $$\EE(X|Y)=\EE(X|Z),\,\,a.s.$$
\end{theorem}

\begin{proof}
 Let $\G=\sigma(Y)=\sigma(Z)$. For any $B\in\G$, we have

 $$ \int_B X\d P=\int_B \EE(X|Y)\d P=\int_B \EE(X|Z)\d P.$$

 Let $T=\EE(X|Y)-\EE(X|Z)$, we have

 $$\int_B T\d P = 0,\,\,\forall B\in\G.$$

 We will prove that $T=0 \,\,a.s.$ For any $\epsilon>0$,

 \begin{align*}
  0\ge\epsilon P(T\ge \epsilon) =\int_{T\ge\epsilon}\epsilon\d P\ge \int_{T\ge\epsilon}X\d P=0.
 \end{align*}

 The last equality holds since $\{T\ge\epsilon\}\in\G$. Hence, $P(T\ge \epsilon)=0$. Similarly, $P(T\le -\epsilon)=0$. Therefore,

 $$P(-\epsilon < T <\epsilon) = P(|T|<\epsilon) = 1.$$

 By letting $\epsilon\to 0$, we conclude that $T=0\,\,a.s.$ Thus, $\EE(X|Y)=\EE(X|Z),\,\,a.s.$
\end{proof}

% \begin{theorem}
%   Let $(\Omega, \F, \PP)$ be a probability space. If $X,Y$ are integrable real-valued random variables, then
%   $\EE[XY] = \EE[X]\EE[Y]$.
% \end{theorem}
% \begin{proof}

% \end{proof}